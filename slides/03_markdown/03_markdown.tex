\documentclass[aspectratio=169]{beamer}
\usepackage{spc}
\usepackage{upquote}
\newcommand\latex{La\h{-0.3ex}TeX}
\begin{document}

\begin{frame}
  \title{\darkblue 3 ~ Markdown}
  \author{\darkgray\bf Arni Magnusson\\
    \phantom{.}\h{32ex}\includegraphics[width=1cm]{github_logo}}
  \date{\darkgreen SPC Git/GitHub Workshop\\[0.5ex]
    Nouméa, 12 November 2025}
  \titlepage
\end{frame}

% ______________________________________________________________________________

\begin{frame}{Overview}
  \begin{itemize}
    \item[] \hyperlink{languages}{\bf\darkblue Markup languages}
    \comment{\latex, HTML, Markdown}\\[4ex]
    \item[] \hyperlink{flavors}{\bf\darkblue Markdown flavors}
    \comment{GitHub, R Markdown, Quarto}\\[4ex]
    \item[] \hyperlink{text}{\bf\darkblue Formatting text}
    \comment{heading, bold, italic, list,\\[0.6ex]
      \h{20.4ex} quote, linebreak, horizontal line, code, link}\\[4ex]
    \item[] \hyperlink{other}{\bf\darkblue Other elements}
    \comment{table, figure, GitHub user/issue/commit}\\[1ex]
  \end{itemize}
\end{frame}

% ______________________________________________________________________________

\begin{frame}\Large
  \hypertarget{languages}
  \centering\darkgreen\bf
  Markup Languages
\end{frame}

% ______________________________________________________________________________

\begin{frame}{Markup Languages}\small
  \textbf{\orange \latex}\\[-1ex]
  \begin{itemize}
    \item[] Introduced: 1978/1985\\[-1ex]
    \item[] Original purpose: books, articles, reports, slides\\[-1ex]
    \item[] Today: can be used directly or as an engine to produce PDF\\[2.5ex]
  \end{itemize}
  \textbf{\blue HTML}\\[-1ex]
  \begin{itemize}
    \item[] Introduced: 1991\\[-1ex]
    \item[] Original purpose: webpages\\[-1ex]
    \item[] Today: paperless format, less restrictive than the `paper metaphor'
    PDF format\\[2.5ex]
  \end{itemize}
  \textbf{\green Markdown}\\[-1ex]
  \begin{itemize}
    \item[] Introduced: 2004\\[-1ex]
    \item[] Original purpose: easy-to-write, easy-to-read, optionally convert to
    HTML\\[-1ex]
    \item[] Today: can produce HTML or PDF (via \latex), can run R code\\[1.5ex]
  \end{itemize}
\end{frame}

% ______________________________________________________________________________

\begin{frame}[fragile]{Markup Languages}\small
  \textbf{\orange \latex: file.tex}
  \vspace{-1.6ex}
  \begin{fns}
    \begin{verbatim}
      \documentclass{article}
      \begin{document}
      Hello world
      \end{document}
    \end{verbatim}
  \end{fns}
  \vspace{-2.3ex}
  \textbf{\blue HTML: file.html}
  \vspace{-1.6ex}
  \begin{fns}
    \begin{verbatim}
      <!DOCTYPE html>
      <title></title>
      <body>
      <p>Hello world
    \end{verbatim}
  \end{fns}
  \vspace{-2.3ex}
  \textbf{\green Markdown: file.md}
  \vspace{-1.6ex}
  \begin{fns}
    \begin{verbatim}
      Hello world
    \end{verbatim}
  \end{fns}
  \vspace{4ex}
\end{frame}

% ______________________________________________________________________________

\begin{frame}\Large
  \hypertarget{flavors}
  \centering\darkgreen\bf
  Markdown Flavors
\end{frame}
% ______________________________________________________________________________

\begin{frame}{Markdown Flavors}\small
  \textbf{\orange GitHub Flavored Markdown}
  \begin{itemize}
    \item[] Features: table, autolink, user/issue/commit
    \item[] Usefulness at SPC: README.md, GitHub discussions\\[3ex]
  \end{itemize}
  \textbf{\blue R Markdown}
  \begin{itemize}
    \item[] Features: run R code
    \item[] Usefulness as SPC: R analyses, HTML/PDF documents\\[3ex]
  \end{itemize}
  \textbf{\green Quarto}
  \begin{itemize}
    \item[] Features: run R and Python code
    \item[] Usefulness as SPC: R analyses, HTML/PDF documents\\[3ex]
  \end{itemize}
\end{frame}

% ______________________________________________________________________________

\begin{frame}\Large
  \hypertarget{text}
  \centering\darkgreen\bf
  Formatting Text
\end{frame}

% ______________________________________________________________________________

\begin{frame}[fragile]{Formatting Text}\small
  \begin{verbatim}
    # Heading level 1
    ## Heading level 2
    ### Heading level 3

    **bold**
    *italic*

    * unordered
    * list

    - alternative
    - style

    1. ordered
    2. list
  \end{verbatim}
  \vspace{2ex}
\end{frame}

% ______________________________________________________________________________

\begin{frame}[fragile]{Formatting Text (cont.)}\small
  \begin{verbatim}
    > quoted
    > text

    line\
    break

    ---

    Inline `code`

    ```
    code block
    ```

    https://example.com
    [click here](https://example.com)
  \end{verbatim}
  \vspace{2ex}
\end{frame}

% ______________________________________________________________________________

\begin{frame}\Large
  \hypertarget{other}
  \centering\darkgreen\bf
  Other Elements
\end{frame}

% ______________________________________________________________________________

\begin{frame}[fragile]{Other Elements}\small
  \begin{verbatim}
    | Table | Header |
    | ----- | ------ |
    | Data  | Values |
    | Data  | Values |


    ![](figure.png)

    <img src="figure.png" alt="">


    @user          -  github username
    #1             -  github issue number
    e15ccce        -  github commit sha code
  \end{verbatim}
\end{frame}

% ______________________________________________________________________________

\begin{frame}{Overview}
  \begin{itemize}
    \item[] \hyperlink{languages}{\bf\darkblue Markup languages}
    \comment{\latex, HTML, Markdown}\\[4ex]
    \item[] \hyperlink{flavors}{\bf\darkblue Markdown flavors}
    \comment{GitHub, R Markdown, Quarto}\\[4ex]
    \item[] \hyperlink{text}{\bf\darkblue Formatting text}
    \comment{heading, bold, italic, list,\\[0.6ex]
      \h{20.4ex} quote, linebreak, horizontal line, code, link}\\[4ex]
    \item[] \hyperlink{other}{\bf\darkblue Other elements}
    \comment{table, figure, GitHub user/issue/commit}\\[1ex]
  \end{itemize}
\end{frame}

\end{document}
